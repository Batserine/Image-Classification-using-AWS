\documentclass[../main.tex]{subfiles}
\begin{document}
\part*{Literature Review}
% \section*{Related papers}
Image Recognition is an essential component of various applications, such as facial recognition, object detection, and product recommendation. With the growing demand for image recognition, many researchers have focused on developing image recognition systems using cloud and edge computing. This literature review summarizes and analyzes five research papers that discuss image recognition as a service.

In the paper "Cloud Strategies for Image Recognition," \citet{9244200} discuss various cloud strategies for image recognition. The authors analyze the features of different cloud providers, such as Amazon Web Services, Google Cloud Platform, and Microsoft Azure, and evaluate their suitability for image recognition tasks. The paper emphasizes the importance of cloud computing in image recognition and provides insights into cloud-based image recognition services.

In "Real-Time Object Detection with TensorFlow Model Using Edge Computing Architecture," \citet{9782169} propose a real-time object detection system using a TensorFlow model and edge computing architecture. The paper presents a system architecture that utilizes edge computing to reduce latency and improve response time. The authors evaluate the system's performance using a dataset of real-world images and demonstrate the effectiveness of the system in detecting objects in real-time.

In "The Utilization of Cloud Computing for Facial Expression Recognition using Amazon Web Services," \citet{9297974} discuss the utilization of cloud computing for facial expression recognition. The authors use Amazon Web Services to develop a facial expression recognition system and evaluate the system's performance using a dataset of facial expressions. The paper demonstrates the effectiveness of cloud-based facial expression recognition systems and highlights the potential of such systems in various applications, such as security and entertainment.

In "Object Detection and Recognition using Amazon Rekognition with Boto3," \citet{9776884} discusses object detection and recognition using Amazon Rekognition with Boto3, a software development kit for Python developers. The paper presents an overview of the Amazon Rekognition service and explains how to use the service with Boto3. The author demonstrates the effectiveness of the Amazon Rekognition service in detecting and recognizing objects in real-time, providing a practical guide for developers.

Finally, \citet{8869300} present "An Efficient Real-time Product Recommendation using Facial Sentiment Analysis." The paper proposes an efficient product recommendation system that utilizes facial sentiment analysis to determine a user's emotional state. The system uses the Microsoft Azure Face API to extract facial features and sentiment scores, which are then used to recommend products. The authors evaluate the system's performance using a dataset of user reviews and demonstrate the system's effectiveness in generating relevant product recommendations.

In conclusion, these five research papers provide insights into image recognition as a service using cloud and edge computing. The papers demonstrate the effectiveness of cloud-based and edge-based image recognition systems and provide practical guidance for developers. The papers also highlight the potential of image recognition in various applications, such as facial recognition, object detection, and product recommendation.

Our project aims to create an elastic web application that uses cloud resources to develop an image recognition application. The project involves designing the architecture and implementing RESTful Web Services to provide access to the application. The use of cloud resources to provide scalable and cost-effective solutions is a common theme in the above 5 research papers. The above research papers emphasize the use of cloud-based services, including AWS services, for scalable and cost-effective image recognition solutions.

\clearpage
\end{document}

