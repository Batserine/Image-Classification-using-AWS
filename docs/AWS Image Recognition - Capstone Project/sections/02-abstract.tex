\documentclass[../main.tex]{subfiles}
\begin{document}
\part*{Introduction}
\section*{Problem Statement}
The project aims at building an elastic web application that can automatically scale out and scale in on-demand and cost-effectively by using cloud resources. The resources used were from Amazon Web Services. It is an image recognition application exposed as a Rest Service to the clients to access.

\section*{Objectives}
 The application takes the images and returns the predicted output by the deep learning model by using the AWS resources, an IaaS provider. AWS as an IaaS provider offers a variety of compute, storage and message services. So the tasks involved designing the architecture, implementing RESTful Web Services, a load balancer that scales in and scales out EC2 instances at App Tier according to the demand of the user.
 
\section*{Milestones}
The initial milestones involve
\begin{itemize}
\item To design the architecture
\item To explore the AWS services.
\item Then, Build an Architecture.
\item Make appropriate changes to the flow of architecture.
\item Finally, Testing and Evaluation.
\end{itemize}

\section*{Tools}
\begin{itemize}
    \item AWS services (EC2, SQS, S3, IAM).
    \item AWS Image Recognition library.
    \item Web Services (HTML, CSS, JS).
    \item Testing for resources.
    \item Python/Java for backend of the architecture.
\end{itemize}

\section*{Problems to Address}
As we know, taking in such a large amount of data for training or modelling means AWS can potentially charge for extra storage to address this issue we are planning to do progressive loading or utilize a big data platform.
\end{document}

\clearpage

