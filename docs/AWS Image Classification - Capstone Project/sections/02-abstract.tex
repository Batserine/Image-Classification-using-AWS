\documentclass[../main.tex]{subfiles}
\begin{document}
\part*{Introduction}
\section*{Problem Statement}
The project aims at building an elastic web application that can automatically scale out and scale in on-demand and cost-effectively by using cloud resources. The resources used were from Amazon Web Services. It is an image classification application exposed as a Rest Service to the clients to access.

\section*{Objectives}
 The application takes the images and returns the predicted output using an image classification model through the AWS resources, an IaaS provider. AWS as an IaaS provider offers a variety of compute, storage and message services. So the tasks involved designing the architecture, implementing RESTful Web Services, a load balancer that scales in and scales out EC2 instances at App Tier according to the demand of the user.
 
\section*{Milestones}
The initial milestones involve
\begin{itemize}
% \item To design the System architecture. 
\item Design and build an interactive system for User.
\item Explore and understand the AWS services and improve the infrastructure.
\item Enable AWS Services and run some tests to see storage and computation performance.
% \item Finally, host the application using open-source platforms.
\end{itemize}

\section*{Tools Used}
\begin{itemize}
    \item AWS services (EC2, SQS, S3).
    \item Pretrained Image classification model.
    \item Web Services (HTML, CSS, Flask).
    \item Testing for resources.
    \item Python for backend of the architecture.
    % \item Heroku for hosting the application.
\end{itemize}

% \section*{Problems to Address}
% \item As we know, taking in such a large amount of data for training or modelling means AWS can potentially charge for extra storage to address this issue we are planning to do progressive loading or utilize a big data platform.

% \item Need to run some tests at the end to make sure the application is running properly on the platform.
\end{document}

\clearpage

